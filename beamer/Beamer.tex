\documentclass[aspectratio=1210]{beamer}

\input{.setup.tex}

\title{Praxissemester}
\author{Enrico Stemmer}

\begin{document}

\begin{frame}
    \titlepage
    \customnote{*Test*}
\end{frame}

\begin{frame}
    \frametitle{Übersicht}
    \tableofcontents
\end{frame}

\section{Einleitung}
\subsection{Vorstellung}
\begin{frame}[fragile]{\insertsubsection}
    \begin{itemize}
        \item<1-2> \textbf{Name:} Enrico Stemmer
        \item<1-2> \textbf{Beruf:} Dualer Student bei ARCA-Consult (Firmeninterne IT)
              \customnote<1>{Beratung für IT Security und Informationssicherheit.}
              \customnote<1>{Viel normen für andere firmen}
              \customnote<1>{Aber auch customizing und aufsetzen von ISMS und IAM Systemen.}
              \vspace{0.2cm}
        \item<1-2> Zeitraum: 01.08.2025 - 18.12.2025
              \vspace{0.2cm}
        \item<2> Präsentation auf \href{https://github.com/h3rmt-thi/praxis}{\texttt{https://github.com/h3rmt-thi/praxis}}
              \customnote<2>{Präsentation mit Latex Beamer und Nix ans Buildsystem für Builden, Präsentieren usw.}
              \customnote<2>{Fragen zwischendurch gerne stellen.}
    \end{itemize}
\end{frame}

\subsection{ARCA-Consult}
\begin{frame}{\insertsubsection}
    \begin{tikzpicture}[remember picture,overlay]
        \node[anchor=north east, xshift=-0.2cm, yshift=-0.2cm] at (current page.north east) {
            \includegraphics[width=2cm]{arca.png}
        };
    \end{tikzpicture}
    \begin{itemize}
        \item Spezialisiert auf IT-Sicherheit und Informationssicherheit
        \item Beratung zu Compliance und Normen (ISO 27001, etc.)
              \customnote{Viele Normen, z.B. ISO 27001/2, TISAX, BSI IT-Grundschutz, NIS-2, etc}
        \item Implementierung von Information Security Management Systems \textit{(ISMS)}
              \customnote{ISMS: System zur Verwaltung der Informationssicherheit in einem Unternehmen}
              \customnote{beinhaltet: \textbf{Regeln}, \textbf{Verfahren}, \textbf{Maßnahmen} und Tools}
              \customnote{Informationssicherheit: schützt Daten/Systeme vor unbefugtem Zugriff, Manipulation und Verlust}
        \item Identity Access Management \textit{(IAM)} Lösungen
              \customnote{IAM: Verwaltung von Benutzeridentitäten und Zugriff auf Ressourcen}
        \item Unterstützung für Unternehmen bei IT-Sicherheitsstrategien
    \end{itemize}
\end{frame}

\section{Projekte}
\subsection{Entwickeln eines Prototypen für die Verwaltung von Normen und Maßnahmen}
\begin{frame}{Entwickeln eines Prototypen für die \\ Verwaltung von Normen und Maßnahmen}
    \begin{tikzpicture}[remember picture,overlay]
        \node[anchor=north east, xshift=-0.2cm, yshift=-0.2cm] at (current page.north east) {
            \includegraphics[width=2cm]{arca.png}
        };
    \end{tikzpicture}
\end{frame}

\subsection{Alternative Infrastruktur zu Microsoft Cloud}
\begin{frame}{Alternative Infrastruktur zu \\Microsoft Cloud}
    \begin{tikzpicture}[remember picture,overlay]
        \node[anchor=north east, xshift=-0.2cm, yshift=-0.2cm] at (current page.north east) {
            \includegraphics[width=2cm]{arca.png}
        };
    \end{tikzpicture}
    \begin{onlyenv}<1>
        \begin{itemize}
            \item Kostensteigerungen bei Microsoft 365
            \item Viele "unnötige" Features
                  \customnote<1>{z.B. Copilot, etc.}
            \item Spezifische Workflows lassen sich nur schwer implementieren oder debuggen
        \end{itemize}
    \end{onlyenv}
    \begin{onlyenv}<2-3>
        \begin{itemize}
            \item Ziel: Alternative Infrastruktur aufbauen, um Kosten zu senken und mehr Kontrolle zu haben
            \item Fokus auf Moderne Open-Source/Source-Available-Tools und Selbsthosting
            \item<3> Viel mehr Infrastruktur notwendig als Anfangs gedacht (VPN, Metrics, Backups, DNS, etc.)
        \end{itemize}
    \end{onlyenv}
    \begin{onlyenv}<4>
        \begin{itemize}
            \item Ausgangssituation: Dedicated Server mit Proxmox bei OVH
                  \customnote<4>{Proxmox: Open-Source Virtualisierungsplattform für VM-Management}
                  \customnote<4>{OVH: 64GB RAM, 16 vCPU, 2TB NVMe}
            \item Test-Umgebung mit Windows-VMs für IAM-Deployments
                  \customnote<4>{Genutzt um IAM-Tools zu testen, Kundenumgebungen zu simulieren}
            \item Pritunl als VPN (OpenVPN/WireGuard Web UI)
                  \customnote<4>{Open-Source VPN-Server, basiert auf OpenVPN und WireGuard}
                  \customnote<4>{User müssen zip datei importieren, schlechter linux support, usw.}
        \end{itemize}
    \end{onlyenv}
    \begin{onlyenv}<5-6>
        \begin{itemize}
            \item<5-6> Ausgangssituation: Dedicated Server mit Proxmox bei OVH
            \item<5-6> Test-Umgebung mit Windows-VMs für IAM-Deployments
            \item<5-6> Kein Monitoring, keine Backups, kein ReverseProxy, etc.
                  \customnote<5>{ReverseProxy für \textbf{DNS + SSL}}
            \item<5-6> Pritunl als VPN (OpenVPN/WireGuard Web UI)
                  \vspace{0.8cm}
            \item<6> Nächste Schritte: Weitere Tools evaluieren und Infrastruktur skalieren
            \item<6> Ersetzen von Tools durch bessere Alternativen
        \end{itemize}
    \end{onlyenv}
    \begin{onlyenv}<7>
        \begin{itemize}
            \item<7> \textbf{Keycloak} für Identity Access Management und SSO
                  \customnote<7>{Anbinden an entra ID}
                  \customnote<7>{Erlaubt login in proxmox UI per Microsoft Konto}
                  \vspace{0.2cm}
            \item<7> \textbf{NetBird} als moderne VPN-Alternative
                  \begin{itemize}
                      \item SSO-Integration und benutzerfreundliche Authentifizierung
                      \item WireGuard-basiert für bessere Performance
                      \item Granulare Zugriffskontrolle und Netzwerk-Segmentierung
                            \customnote<7>{Dev server / Prod server, nur IT zugriff auf Dev services, etc.}
                  \end{itemize}
                  \vspace{0.2cm}
            \item<9> \textbf{Caddy}/\textbf{Traefik} als Reverse Proxy und Load Balancer
                  \vspace{0.2cm}
            \item<9> \textbf{Prometheus}, \textbf{Loki}, \textbf{Tempo} + \textbf{Grafana} für Monitoring und Logging
                  \vspace{0.2cm}
            \item<13> \textbf{Ansible} für automatisierte, konsistente Deployments und Konfigurationsverwaltung
            \item<13> \textbf{Semaphore UI} für vereinfachte Verwaltung und Ausführung von Ansible Playbooks      \end{itemize}
    \end{onlyenv}
    \begin{onlyenv}<8>
        \begin{tikzpicture}[remember picture,overlay]
            \node[anchor=center, yshift=-0.4cm] at (current page.center) {
                \includegraphics[width=\textwidth,keepaspectratio]{netbird.png}
            };
        \end{tikzpicture}
    \end{onlyenv}
    \begin{onlyenv}<9>
        \begin{itemize}
            \item<9> \textbf{Keycloak} für Identity Access Management und SSO
                  \vspace{0.2cm}
            \item<9> \textbf{NetBird} als moderne VPN-Alternative
                  \begin{itemize}
                      \item SSO-Integration und benutzerfreundliche Authentifizierung
                      \item WireGuard-basiert für bessere Performance
                      \item Granulare Zugriffskontrolle und Netzwerk-Segmentierung
                  \end{itemize}
                  \vspace{0.2cm}
            \item<9> \textbf{Caddy}/\textbf{Traefik} als Reverse Proxy und Load Balancer
                  \vspace{0.2cm}
            \item<9> \textbf{Prometheus}, \textbf{Loki}, \textbf{Tempo} + \textbf{Grafana} für Monitoring und Logging
                  \customnote<9>{Grafana auch an SSO angebunden}
                  \vspace{0.2cm}
            \item<13> \textbf{Ansible} für automatisierte, konsistente Deployments und Konfigurationsverwaltung
            \item<13> \textbf{Semaphore UI} für vereinfachte Verwaltung und Ausführung von Ansible Playbooks      \end{itemize}
    \end{onlyenv}
    \begin{onlyenv}<10>
        \begin{tikzpicture}[remember picture,overlay]
            \node[anchor=center, yshift=-0.4cm] at (current page.center) {
                \includegraphics[width=1.05\textwidth,keepaspectratio]{grafana.png}
            };
        \end{tikzpicture}
    \end{onlyenv}
    \begin{onlyenv}<11>
        \begin{tikzpicture}[remember picture,overlay]
            \node[anchor=center, yshift=-0.4cm] at (current page.center) {
                \includegraphics[width=1.05\textwidth,keepaspectratio]{logs.png}
            };
        \end{tikzpicture}
    \end{onlyenv}
    \begin{onlyenv}<12>
        \begin{tikzpicture}[remember picture,overlay]
            \node[anchor=center, yshift=-0.4cm] at (current page.center) {
                \includegraphics[width=1.05\textwidth,keepaspectratio]{traces.png}
            };
        \end{tikzpicture}
    \end{onlyenv}
    \begin{onlyenv}<13>
        \begin{itemize}
            \item<13> \textbf{Keycloak} für Identity Access Management und SSO
                  \vspace{0.2cm}
            \item<13> \textbf{NetBird} als moderne VPN-Alternative
                  \begin{itemize}
                      \item SSO-Integration und benutzerfreundliche Authentifizierung
                      \item WireGuard-basiert für bessere Performance
                      \item Granulare Zugriffskontrolle und Netzwerk-Segmentierung
                  \end{itemize}
                  \vspace{0.2cm}
            \item<13> \textbf{Caddy}/\textbf{Traefik} als Reverse Proxy und Load Balancer
                  \vspace{0.2cm}
            \item<13> \textbf{Prometheus}, \textbf{Loki}, \textbf{Tempo} + \textbf{Grafana} für Monitoring und Logging
                  \vspace{0.2cm}
            \item<13> \textbf{Ansible} für automatisierte, konsistente Deployments und Konfigurationsverwaltung
            \item<13> \textbf{Semaphore UI} für vereinfachte Verwaltung und Ausführung von Ansible Playbooks
                  \customnote<13>{Einfach taskausführung über WebUI, Login per SSO}
        \end{itemize}
    \end{onlyenv}
\end{frame}

\end{document}