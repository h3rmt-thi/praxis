\documentclass[aspectratio=1210]{beamer}

\input{.setup.tex}

\title{Praxissemester}
\author{Enrico Stemmer}

\begin{document}

\begin{frame}
	\titlepage
	\customnote{*Test*}
\end{frame}

\begin{frame}
	\frametitle{Übersicht}
	\tableofcontents
\end{frame}

\section{Einleitung}
\subsection{Vorstellung}
\begin{frame}{\insertsubsection}
	\begin{tikzpicture}[remember picture,overlay]
		\node[anchor=north east, xshift=-0.2cm, yshift=-0.2cm] at (current page.north east) {
			\includegraphics[width=2cm]{arca.png}
		};
	\end{tikzpicture}
	\begin{itemize}
		\item \textbf{Name:} Enrico Stemmer
		\item \textbf{Beruf:} Dualer student bei ARCA-Consult (Firmeninterne IT)
		      \customnote{Beratung für IT Security und Informationssicherheit.}
			  \customnote{Viel normen für andere firmen}
			  \customnote{Aber auch customizing und aufsetzen von ISMS und IAM Systemen.}
		\item Zeitraum: 01.08.2025 - 18.12.2025
		      \vspace{0.5cm}
		\item Präsentation auf \href{https://github.com/h3rmt-thi/praxis}{\texttt{https://github.com/h3rmt-thi/praxis}}
		      \customnote{Fragen zwischendurch gerne stellen.}
	\end{itemize}
\end{frame}

\subsection{Einführung}
\begin{frame}{\insertsubsection}
	\begin{tikzpicture}[remember picture,overlay]
		\node[anchor=north east, xshift=-0.2cm, yshift=-0.2cm] at (current page.north east) {
			\includegraphics[width=2cm]{arca.png}
		};
	\end{tikzpicture}
	\begin{itemize}
		\item 6 Monate, 24 ECTS
		\item Praktische Tätigkeit in einem Unternehmen
		\item Anfertigung einer wissenschaftlichen Arbeit
	\end{itemize}

\end{frame}

\end{document}