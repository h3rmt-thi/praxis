\documentclass[aspectratio=1210]{beamer}

\input{.setup.tex}

\title{Ausnutzung von Prozessorlücken}
\author{Enrico Stemmer}

\begin{document}

\begin{frame}
	\titlepage
	\customnote{*Test* Markdown}
\end{frame}

\begin{frame}
	\frametitle{Übersicht}
	\tableofcontents
\end{frame}

\section{Einleitung}
\subsection{Vorstellung}
\begin{frame}{\insertsubsection}
	\begin{itemize}
		\item \textbf{Name:} Enrico Stemmer
		\item \textbf{Beruf:} Dualer student bei ARCA-Consult (Softwareentwickler)
		      \customnote{Beratung für IT Security und Informationssicherheit.}
		      \vspace{0.5cm}
		\item Präsentation auf \href{https://github.com/h3rmt-thi/praxis}{\texttt{https://github.com/h3rmt-thi/praxis}}
		      \customnote{Fragen zwischendurch gerne stellen.}
	\end{itemize}
\end{frame}

\subsection{Einführung}
\begin{frame}{\insertsubsection\ (Was sind Prozessorlücken?)}
	\begin{itemize}
		\item \textbf{Ziel:} Arbeitsspeicher auslesen
		\item Speicherisolation durch Betriebssystem wird umgangen
		      \customnote{*OS* separiert Speicherbereiche, damit Programme nicht aufeinander zugreifen können. MMU (Memory Management Unit)}
		      \customnote{Browser: V8 JavaScript Isolates}
		\item Zugriff auf fremden Speicher (Programme, Kernel, VMs)
		      \customnote{Cloudcomputing als angriffsziel, weil viele VMs auf einem Server}
		\item Es muss bereits Code auf dem Zielsystem ausgeführt werden
		      \customnote{aber auch JS auf webiste an sich code execution}
		      \customnote{gibt remote spectre}
	\end{itemize}
\end{frame}

\end{document}